%==============================================================================
% Template research proposal bachelor thesis
%==============================================================================%
% Compile in TeXstudio:
%
% - Make sure Biber compiles the bibliography (not Biblatex)
%   Options > Configure > Build > Default Bibliography Tool: "txs:///biber"
% - F5 to compile and watch the result
% - If the bibliography is not visible, try F5 - F8 - F5
%   F8 compiles the bibliography separately
%
% when using JabRef for the bibliography, make sure
% you save in ``biblatex''-mode: File > Switch to BibLaTeX mode.

\documentclass{hogent-article}

\usepackage{lipsum} % Voor vultekst
%------------------------------------------------------------------------------
% Metadata on the article
%------------------------------------------------------------------------------

%---------- Title & author ----------------------------------------------------

% TODO: (phase 2) choose a working title for your proposal
\PaperTitle{Title of the article}
% Typically this is the assignment and the subject you are writing this article for.
% e.g. ``Report research project research techniques 2021-2022''
\PaperType{Paper Research Methods: Research proposal}

% TODO: (phase 1) fill in your own name as author, supply your email address!
\Authors{Steven Stevens\textsuperscript{1}, Jan Janssens\textsuperscript{2}} % Authors

% If this is really a proposal for your bachelor thesis, you are required to write the name of your %co-promotor here. If not, you can leave it blank. 
\CoPromotor{}

% Contact info: Write down contact details of every author of the article, 
% if applicable, also the co-promotor.
\affiliation{
  \textsuperscript{1} \href{mailto:steven.stevens.u1234@student.hogent.be}{steven.stevens.u1234@student.hogent.be}}
\affiliation{
  \textsuperscript{2} \href{mailto:jan.janssens.u4321@student.hogent.be}{mailto:jan.janssens.u4321@student.hogent.be}
}

%---------- Abstract ----------------------------------------------------------

\Abstract{% TODO: (fase 6)
Here you write the abstract of your article, as a continuous text that is one paragraph long. 

As keywords provide the research area (the main domain, your study specialisation) together with other keywords that describe your work best. 
}

%---------- Research area and keywords----------------------------------------
% TODO: (phase 2) Add and/or adjust the keywords.

% The first keyword describes the research area. Choose from this list:
%  
% - Mobile application development
% - Web application development
% - Application development (other)
% - System management
% - Network management
% - Mainframe
% - E-business
% - Databases and big data
% - Machine learning techniques and artificial intelligence 
% - Other (specify)
%
% The other keywords are free of choice.

\Keywords{Research area; Keyword1; Keyword2; Keyword3}
\newcommand{\keywordname}{Keywords} % Defines the keywords heading name

%---------- Title, content -----------------------------------------------------

\begin{document}

\flushbottom % Makes all text pages the same height
\maketitle % Print the title and abstract box
\tableofcontents % Print the contents section
\thispagestyle{empty} % Removes page numbering from the first page

%------------------------------------------------------------------------------
% Main text 
%------------------------------------------------------------------------------

\section{Introduction}

% TODO: (phase 2) Introduce your chosen subject, formulate the research question and subquestions. What is the goal (is this defined in a S.M.A.R.T. way?), what will be the result of  your research? (a Proof-of-Concept, a prototype, an advice, ...)? Why is it useful to research this subject? 

\lipsum[1-3]

\section{Literature overview}

% TODO: (phase 4) write the literature study and use references where appropriate.

% Referencing literature:
% \autocite{BIBTEXKEY} -> (Author, year)
% \textcite{BIBTEXKEY} -> Author (year)
Example of a reference where the authorname is not a part of the sentence is ~\autocite{Moore2002}.

\lipsum[4-9]

\section{Methodology}

% TODO: (phase 5) Describe in detail in what phases your research consists of. How long will each phase take and what will be the result of echt phase? What research techniques are your going to apply of answer each of your research questions? Are you using experiments, questionnaires, simulations? Describe what tools you expect to be using or developing to conduct your research. 

\lipsum[10-12]

\section{Expected conclusions}

% TODO: (phase 6) Describe what you expect as result of your research and why? (e.g. literature review indicates software package A is most commonly used, and seems to be the best option for this case too because of the requirements or needs.) Of course it is impossible to predict the future, so you shouldn't exclude any alternative options yet. In practice, research leads to surprising conclusions, making it even more interesting!

\lipsum[14-18]

%------------------------------------------------------------------------------
% Reference list
%------------------------------------------------------------------------------
% TODO: (phase 4) the referenced items should be included in your BibTeX file
% bibliografie.bib. Use JabRef to keep your bibliography up to date.

\phantomsection
\printbibliography[heading=bibintoc]

\end{document}
